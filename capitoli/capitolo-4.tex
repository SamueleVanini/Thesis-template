% !TEX encoding = UTF-8
% !TEX TS-program = pdflatex
% !TEX root = ../tesi.tex

%**************************************************************
\chapter{Valutazione retrospettiva e conclusioni}
\label{cap:conclusioni}
%**************************************************************

\intro{Nel corso di questo capitolo si presenta una valutazione retrospettiva sullo stage.
Vengono descritti gli obiettivi raggiunti, i progressi personali e il gap tra la formazione
offerta dall'Università e quella richiesta dallo stage.}

%**************************************************************
\section{Obiettivi raggiunti}
Terminato lo stage posso fare un bilancio del grado di soddisfacimento raggiunto negli obiettivi 
personali e aziendali fissati precedentemente, nello specifico nelle sezioni \ref{sec:obiettivi-cap-2} e \ref{sub-sub-sec:obiettivi-cap-3}.
Dal punto di vista aziendale, ho pienamente soddisfatto tutti gli obiettivi obbligatori e facolativi, nonostante le varie difficoltà riscontrate,
sviluppando un prodotto che si può definire completo e che rispetta tutte le specifiche datemi. Oltre a questo ho permesso all'azienda
di approfondire nuove tecnologie utilizzabili per lo sviluppo dei loro prodotti, portandone alla luce pregi e difetti.
Dal punto di vista personale, invece, sono molto soddisfatto del percorso intrapreso. Questo ha rispecchiato completamente le mie
aspettative, dandomi numerose sfide e opportunità di crescita come sviluppatore. Un resoconto  dettagliato della mia progressione
è riportato nella sezione \ref{sec:prog-personale}.

\section{Difficoltà riscontrate}
Le difficoltà riscontrate durante lo sviluppo del prodotto sono state numerose e di varia natura. Escludendo quelle previste,
come lo studio autonomo delle tecnologie da impiegare o il funzionamento della tecnologia RFID, ritengo che il fattore maggiormente
penalizante sia stato la documentazione offerta dal produttore del lettore. Questa non è mai stata intesa come un valido strumento di aiuto agli sviluppatori,
in quanto il produttore stesso oltre ai \emph{devise} offre delle sue soluzione proprietarie per il funzionamento e la gestione di essi.
A fini commerciali quindi, la documentazione è risultata considerevolmente scarsa ed ostica.
Questo, unito alla presenza di alcuni bug o incompletezze che ho riscontrato all'interno del loro codice ha portato ad una frequente comunicazione fra me ed il loro reparto di sviluppo.
Questo necessario scambio di informazioni ha invevitabilmente ridotto il tempo a mia disposizione, considerato adeguato in qualsiasi altra circostanza, per soddisfare le richieste fattemi.

\section{Progressione personale}
\label{sec:prog-personale}
Questo stage aveva molteplici obiettivi formativi, da quelli prettamente legati alle tecnologie che avrei dovuto imparare ad usare fino ad arrivare all’integrazione 
nell’ambiente aziendale, seguendo il metodo di lavoro utilizzato.
\begin{itemize}
    \item \textbf{Personale}: le conoscenze acquisite durante questo stage sono state sicuramente di grande rilevanza. Grazie alla completa immersione
    nell'ambiente lavorativo ho potuto consolidare maggiormente quanto la comunicazione e la collaborazione ricoprano un ruolo fondamentale per la buona riuscita di ogni progetto.
    Oltre a ciò, la possibilità di stare a stretto contatto con numerosi professionisti, mi ha permesso di toccare con mano il processo con cui avviene la risoluzione un problema complesso.
    Poter apprendere una moltitudine di \emph{best practice} relative allo sviluppo è stato non solo interessante ma utile in risvolti pratici in maniera non quantificabile.
    \item \textbf{Tecnico}: dal punto di vista tecnico invece posso dire di aver appreso molto, sia rafforzando le conoscenze derivanti dal corso di studi, 
    sia acquisendone di nuove, relative alle tecnologie utilizzate.
    L’azienda infatti mi ha permesso, tramite questo stage, di ampliare il mio bagaglio di conoscenze permettendomi di interfacciarmi con tali tecnologie, 
    lasciandomi il tempo necessario, per imparare ed implementare al meglio ciò su cui mi stavo formando.
    \begin{itemize}
        \item \textbf{Conoscenza dei \emph{Framework} OSGi}: lo stage mi ha permesso di venire a conoscenza di questa tecnologia per lo sviluppo di soluzioni di medie e grandi dimensioni,
        dandomi la possibilità non solo di approfondirla ma di poter interfacciarmi con molto del codice sviluppato da loro proprio con questa tecnologia.
        \item \textbf{Sviluppo con la libreria JNA}: per lo sviluppo del prodotto è stato inevitabile l'utilizzo della libreria JNA. Questo mi ha dato modo di poter consolidare
        ulteriormente le mie conoscenza riguardo al codice nativo ed al funzionamento dell'ecosistema JVM, apprese durante gli anni di studi, garantendomi allo stesso tempo
        un elevato grado di approfondimento su tali materie.
        \item \textbf{Conoscenza della tecnolgia RFID}: essendo tutto lo stage incentrato su tale tecnologia, ho avuto modo di studiarne il funziomanto nel dettaglio, garantendomi non solo
        una comprensione generale, ma la capacità di applicarne le nozioni in contesti e situazioni reali.
    \end{itemize}
\end{itemize}

\section{Gap formativo tra Università e stage}
L'ambiente lavorativo che ho trovato, seppur discostandosi in modo significativo da quello universitario, non ne risulta completamente estraneo, anzi, sono molti i punti di contatto fra questi.
Nonostante il percoso di studi abbia numerosi corsi con il fine ultimo di insegnare una materia propedeutica alla programmazione, non è possibile non vedere le similitudini
con gli ultimi corsi intrapresi nell'ultimo anno di studi. A mio parere questi sono risultati particolarmente utili per affrontare lo stage con il corretto \emph{mindset} e un solido \emph{way of working}.
Sono state molteplici le volte in cui non ho trovato grandi differenze tra il progetto intrapreso durante il corso di Ingneria del Software, e le varie fasi affrontate durante la realizzazione
del progetto di stage. Penso non sia concepibile un corso di studi in cui vengono insegnate solo le ultime tecnologie in voga sul mercato. 
Trovo invece che il bilanciamento offerto dia ad ogni studente tutti gli strumenti necessari per un apprendimento rapido e continuo.
Comprendere una solida formazione di base ed un progetto avente lo scopo di realizzare un prodotto completo, è sicuramente una buona scelta per costruire un valido bagaglio di conoscenze.
Tuttavia ritengo che questo bilanciamento sia ulteriormente migliorabile dando una maggiore continuità ai vari corsi di studi e un approfondimento maggiore ad alcuni di essi. 
In molte occassioni mi è sembrato che ci fossero delle importanti mancanze su come un corso si colleggasse ad un altro. In aggiunta proporrei l'introduzione di alcuni progetti che vadano a comprendere corsi affini.
Questo seppur dia l'impressione di essere fine a se stesso garantirebbe ad ogni studente la possibilità di mettere in pratica quanto imparato formando allo stesso tempo un senso di continuità e completezza alle materie
insegnate. 

%
%\section{Casi d'uso}

%Per lo studio dei casi di utilizzo del prodotto sono stati creati dei diagrammi.
%I diagrammi dei casi d'uso (in inglese \emph{Use Case Diagram}) sono diagrammi di tipo \gls{uml} dedicati alla descrizione delle funzioni o servizi offerti da un sistema, così come sono percepiti e utilizzati dagli attori che interagiscono col sistema stesso.
%Essendo il progetto finalizzato alla creazione di un tool per l'automazione di un processo, le interazioni da parte dell'utilizzatore devono essere ovviamente ridotte allo stretto necessario. Per questo motivo i diagrammi d'uso risultano semplici e in numero ridotto.

%\begin{figure}[!h] 
%    \centering 
%    \includegraphics[width=0.9\columnwidth]{usecase/scenario-principale} 
%    \caption{Use Case - UC0: Scenario principale}
%\end{figure}

%\begin{usecase}{0}{Scenario principale}
%\usecaseactors{Sviluppatore applicativi}
%\usecasepre{Lo sviluppatore è entrato nel plug-in di simulazione all'interno dell'IDE}
%\usecasedesc{La finestra di simulazione mette a disposizione i comandi per configurare, registrare o eseguire un test}
%\usecasepost{Il sistema è pronto per permettere una nuova interazione}
%\label{uc:scenario-principale}
%\end{usecase}

%\section{Tracciamento dei requisiti}

%Da un'attenta analisi dei requisiti e degli use case effettuata sul progetto è stata stilata la tabella che traccia i requisiti in rapporto agli use case.\\
%Sono stati individuati diversi tipi di requisiti e si è quindi fatto utilizzo di un codice identificativo per distinguerli.\\
%Il codice dei requisiti è così strutturato R(F/Q/V)(N/D/O) dove:
%\begin{enumerate}
%	\item[R =] requisito
%    \item[F =] funzionale
%    \item[Q =] qualitativo
%    \item[V =] di vincolo
%    \item[N =] obbligatorio (necessario)
%    \item[D =] desiderabile
%    \item[Z =] opzionale
%\end{enumerate}
%Nelle tabelle \ref{tab:requisiti-funzionali}, \ref{tab:requisiti-qualitativi} e \ref{tab:requisiti-vincolo} sono riassunti i requisiti e il loro tracciamento con gli use case delineati in fase di analisi.

%\newpage

%\begin{table}%
%\caption{Tabella del tracciamento dei requisti funzionali}
%\label{tab:requisiti-funzionali}
%\begin{tabularx}{\textwidth}{lXl}
%\hline\hline
%\textbf{Requisito} & \textbf{Descrizione} & \textbf{Use Case}\\
%\hline
%RFN-1     & L'interfaccia permette di configurare il tipo di sonde del test & UC1 \\
%\hline
%\end{tabularx}
%\end{table}%

%\begin{table}%
%\caption{Tabella del tracciamento dei requisiti qualitativi}
%\label{tab:requisiti-qualitativi}
%\begin{tabularx}{\textwidth}{lXl}
%\hline\hline
%\textbf{Requisito} & \textbf{Descrizione} & \textbf{Use Case}\\
%\hline
%RQD-1    & Le prestazioni del simulatore hardware deve garantire la giusta esecuzione dei test e non la generazione di falsi negativi & - \\
%\hline
%\end{tabularx}
%\end{table}%

%\begin{table}%
%\caption{Tabella del tracciamento dei requisiti di vincolo}
%\label{tab:requisiti-vincolo}
%\begin{tabularx}{\textwidth}{lXl}
%\hline\hline
%\textbf{Requisito} & \textbf{Descrizione} & \textbf{Use Case}\\
%\hline
%RVO-1    & La libreria per l'esecuzione dei test automatici deve essere riutilizzabile & - \\
%\hline
%\end{tabularx}
%\end{table}%