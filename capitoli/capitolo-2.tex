% !TEX encoding = UTF-8
% !TEX TS-program = pdflatex
% !TEX root = ../tesi.tex

%**************************************************************
\chapter{Il progetto Kathrein}
\label{cap:lo-stage}
%**************************************************************

\intro{Nel corso di questo capitolo si presenterà il progetto di stage a cui ho partecipato andando ad esporre come questo sia interpretato dall'azienda.
Ne verranno poi descritti gli obiettivi, i vincoli ed il futuro una volta terminata la mia partecipazione}\\

%**************************************************************
\section{Descrizione}
Il progetto di stage propostomi da Aton S.P.A., mediante la figura del mio \emph{tutor} interno, è riconducibile alla visione d'insieme che l'azienda possiede
nei confronti delle collaborazioni con gli atenei universitari. Tali percorsi vengono interpretati come mezzo per ampliare il bagaglio di conoscenze proprio 
dell'azienda riguardo a nuove tecnologie. Questo scopo viene perseguito tramite l'implementazione di \emph{software} che possa contribuire alle soluzioni 
proprietarie utlizzate poi nei vari progetti cliente. Oltre a questo, gli stage proprosti si pongono come secondo fine la pubblicizzazione dell'azienda 
cercando di attirare nuovi sviluppatori che possano far crescere il team di programmatori già esistente. \\
Nel mio caso specifico lo scopo del progetto era quello di studiare l'interazione con dei nuovi lettori RFID al fine di poterne sfruttare ogni funzionalità 
tramite l'integrazione con la piattaforma proprietaria già in uso preso il reparto RFID. \\
Le richieste principali sono state:
\begin{itemize}
    \item Funzionamento autono: dovendo integrare questi lettori in contensti industriali in cui il carico di lavoro risulta essere particolarmente elevato
    era necessario che i dispositivi potessero lavorare senza la presenza di un operatore umano che gli azionasse e controllasse;
    \item Tolleranza ai guasti: operando in contesti in cui agenti fisici possono facilmente interrompere la connessione ai dispositivi era necesserio
    identificare e gestire tali anomalie;
    \item Integrazione con piattaforme esistenti: dovendo cooperare con piattaforme e software proprietari era necessario
\end{itemize} 
%**************************************************************
\section{Obiettivi}
\subsection{Obiettivi aziendali}
\subsection{Obiettivi personali}

%**************************************************************
\section{Vincoli}
\subsection{Vincoli tecnologici}
\subsection{Vincoli temporali}

%**************************************************************
\section{Futuro del progetto}
