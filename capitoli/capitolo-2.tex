% !TEX encoding = UTF-8
% !TEX TS-program = pdflatex
% !TEX root = ../tesi.tex

%**************************************************************
\chapter{Il progetto Kathrein}
\label{cap:lo-stage}
%**************************************************************

\intro{Nel corso di questo capitolo si presenta il progetto di stage esponendo come queste attività siano interpretate dall'azienda.
Ne vengono poi descritti gli obiettivi, i vincoli ed il futuro al termine della mia partecipazione.}

%**************************************************************
\section{Descrizione}
\label{sec:descrizione}
Il progetto di stage propostomi da Aton S.P.A., mediante la figura del mio \emph{tutor} interno, è riconducibile alla visione d'insieme che l'azienda possiede
nei confronti delle collaborazioni con le università. Queste opportunità non sono nuove all'azienda, nel corso degli anni sono state svolte
numerose collaborazioni con diversi atenei che hanno portato alla realizzazione di prodotti ancora oggi utilizzati. \\
Tali percorsi vengono interpretati come mezzo per ampliare il bagaglio di conoscenze proprio dell'azienda riguardo a nuove tecnologie.
Questo scopo viene perseguito tramite l'implementazione di \emph{software} che possa contribuire alle soluzioni 
utilizzate nei vari progetti cliente. Oltre a questo, gli stage proprosti si pongono come secondo fine la pubblicizzazione dell'azienda 
cercando di attirare nuovi sviluppatori che possano far crescere il team già esistente. \\
Nel mio caso specifico lo scopo del progetto era quello di studiare l'interazione con dei nuovi lettori RFID per poi applicare quanto imparato tramite lo
sviluppo di un prodotto \emph{software} apposito. Tale prodotto era quindi finalizzato a diventare un nuovo componente utilizzabile dai tecnici aziendali per 
la creazione di flussi di lavoro adoperanti i lettori analizzati, accrescendo di conseguenza le soluzioni offerte dall'azienda nei vari progetti cliente.
Un esempio concreto della posizione di questo prodotto, all'interno di un sistema RFID classico, è rappresentato nei passi 5 e 6, riportati nella figura 
\ref{rfid-system}.


\begin{figure}[!ht] 
    \centering 
    \includegraphics[width=1\columnwidth]{tec/sistema_rfid.jpeg} 
    \caption{Schema di un sistema RFID - \textbf{fonte} \url{urly.it/3gc8k}}
    \label{rfid-system}
\end{figure}

%Le richieste principali sono state:
%\begin{itemize}
%    \item \textbf{Funzionamento autonomo}: dovendo integrare questi lettori in contesti industriali in cui il carico di lavoro risulta essere particolarmente 
%    elevato, era necessario che i dispositivi potessero lavorare senza la presenza di un operatore umano, che gli azionasse e controllasse;
%    \item \textbf{Tolleranza ai guasti}: operando in contesti in cui agenti fisici possono facilmente interrompere la connessione ai dispositivi, era necesserio
%    identificare e gestire tali anomalie;
%    \item \textbf{Integrazione con piattaforme esistenti}: dovendo cooperare con piattaforme e software proprietari, era necessario implementare una strategia
%    di comunicazione con tali piattaforme;
%\end{itemize}

%Nel decidere come affrontare il problema ho vagliato alcune soluzioni.
%\begin{itemize}
%    \item La prima riguardava la possibilità di sviluppare un applicativo a se stante. Questo si sarebbe interfacciato 
%    con il lettore tramite le librerie offerte dal produttore, per poi andare a comunicare con AMP come illustrato in figura \ref{socket_web} tramite l'utilizzo di un \gls{socket_webg}.

%    \begin{figure}[!h] 
%        \centering 
%        \includegraphics[width=1\columnwidth]{tec/socket_web.png} 
%        \caption{Connessione tramite Socket Web - \textbf{fonte} \url{urly.it/3g8w2}}
%        \label{socket_web}
%    \end{figure}

%    Il vantaggio principale di questa soluzione era la possibilità di essere completamente indipendente dal \emph{framework OSGi}, avendo così più 
%    libertà in merito alle tecnologie da utilizzare come la versione Java o le librerie necessarie al \emph{logging}.
%    Il maggior problema consisteva nella gestione del ciclo di vita del modulo, questo sarebbe stato eseguito come indipendente dalla paittaforma principale
%    portando di conseguenza molte problematiche su come mantenere solida l'esperienza utente.
%    Il secondo problema riscontrato era dovuto alla gestione degli errori. Utilizzando questa soluzione si sarebbe andato ad aggiungere un nuovo punto
%    di potenziale errore, ovvero, la comunicazione tra l'applicativo e la piattaforma stessa. In aggiunta, nonostante una gestione ottimale di questa comunicazione,
%    le performance generali ottenute non avrebbero soddisfatto a pieno i requisiti attesi.
    
%    \item Una seconda soluzione invece consisteva nello sviluppo di un modulo per AMP, che si andasse ad interfacciare con il lettore tramite un protocollo
%    testuale \gls{xmlg} di basso livello, chiamato LLRP. 
%    I vantaggi di questa soluzione erano molteplici; come prima cosa vi era una facile gestione del modulo,
%    questo essendo inserito direttamente all'interno di AMP non richiedeva la gestione di un vero e proprio ciclo di vita in quanto già garantito dalla
%    piattaforma stessa. Il secondo grande vantaggio era da ricercarsi nell'utilizzo del protocollo LLRP, questo costituisce uno standard all'interno del
%    settore RFID, garantendo quindi una vasto ecosistema di stumenti e librerie per l'interazione con esso. In aggiunta il protocollo si basa su un'altro 
%    popolare protocollo di rete, ovvero TCP, che garantisce la consegna di ogni pacchetto dati inviato.
%    Nonostante i numerosi vantaggi anche questa soluzione si è dovuta scartare, questo a causa di alcune caratteristiche dei lettori utilizzati che non
%    essendo di fatto standard e di libera licenza non vengono supportate dal protocollo stesso.
%\end{itemize}

%La soluzione adottata è stata l'unione delle due sopra proposta in modo che potesse andare a conciliare i vantaggi di entrambe limitandone al
%contempo le problematiche. Ho scelto quindi di creare un modulo per AMP che andasse però ad utilizzare le librerie proposte dal produttore per interfacciarsi
%con il lettore. Così facendo si ottiene una soluzione dalle molteplici caratteristiche:
%\begin{itemize}
%    \item \textbf{Alte prestazioni}: utilizzando le librerie native offerte dal produttore, e l'integrazione diretta all'interno della piattaforma, non sono presenti
%    colli di bottiglia che possano degradare le prestazioni;
%    \item \textbf{Esperienza utente solida}: l'utente utilizzatore della piattaforma non troverà differenze nell'utilizzo di essa, indifferentemente dal lettore in utilizzo, 
%    rendendo così l'esperienza d'uso più semplice e piacevole; 
%\end{itemize}
%Per la stesura del codice ho quindi optato per l'utilizzo di Java 8 supportano dal framework OSGi Knopflerfish precedentemente descritto. In aggiunta a questo
%ho fatto utilizzo della libreria JNA per interfacciarmi con le librerie offerte dal produttore. JNA è una libreria \emph{open-source} che permette di accedere
%a codice nativo, ovvero scritto in un linguaggio compilato come C e C++, direttamente dal codice scritto in Java. Grazie a questo è possibile realizzare
%soluzioni utilizzando tutti i vantaggi di un linguaggio maturo e di alto livello senza però sacrificare le prestazioni offerte dalle librerie native. \\
%L'obiettivo a cui mira questa soluzione è quello di apprendere le nozioni relative allo sviluppo per i lettori RFID Kathrein tramite le librerie native, cercandone
%quindi i pregi ed i difetti. Inoltre viene sperimentato se l'interfacciamento con codice nativo possa essere una valida alternativa all'utilizzo
%del protocollo LLRP. Se così fosse si potrebbero sfruttare tutte le caratteristiche proprietarie di altri lettori, senza avere la necessità di cambiare strumenti
%di sviluppo, risparmiando di conseguenza molte ore di lavoro e soprattutto abbatendo i costi di formazione per tali tecnologie.


%**************************************************************
\section{Obiettivi}
\subsection{Obiettivi aziendali}
L'azienda ha posto dei macro obiettivi principali per quanto riguarda le funzionalità dell'applicativo, sia a livello utente che a livello software. Questi sono stati
esplicitati nei primi giorni in cui mi è stato presentato il progretto in dettaglio dal mio \emph{tutor} interno.
\begin{itemize}
    \item L'applicativo avrebbe dovuto interfacciarsi con tutte le generazioni di lettori Kathrein;
    \item L'applicativo avrebbe dovuto funzionare in maniera autonoma, senza quindi la presenza di un operatore atto ad operare i \emph{device}; 
    \item L'applicativo avrebbe dovuto offrire le stesse funzionalità all'utente degli altri applicativi, dedicati ad altri marchi di lettori;
    \item L'applicativo avrebbe dovuto supportare la lettura di numerosi \emph{tag} simultaneamente, nell'ordine di qualche centinaio;
    \item L'applicativo avrebbe dovuto essere fruibile direttamente all'interno della piattaforma AMP;
    \item Le anomalie dovute alla rete dovevano essere gestite e notificate se non risolvibili;
    \item L'applicativo avrebbe dovuto comandare altri dispositivi IoT come semafori o \emph{buzzer} il cui operato dipendesse da quello del lettore.
\end{itemize}
\subsection{Obiettivi personali}
Tramite l'Università ho avuto accesso all'evento \textbf{Stage-IT}, offerto da AssIndustria Veneto-Centro, in cui vengono messi in diretto collegamento aziende in collaborazione con
il Dipartimento di Padova e gli studenti laureandi, in modo da poter proporre stage pre-laurea.
Durante l'evento numerosi progetti hanno attratto la mia attenzione, l'offerta proposta da Aton S.P.A. però è riuscita a catturare a pieno il mio interesse.
Causa di questo è stata la possibilità di lavorare in un contesto a me nuovo ma in fortissima crescita nel mercato odierno. In particolare, l'utilizzo di codice nativo
tramite Java, all'interno di un contesto in cui l'ottimizzazione ricopre un ruolo cruciale, mi ha fin da subito affascianto. Questo mi ha permesso di
applicare quanto imparato durante gli anni studio, garantendomi allo stesso tempo una grande occasione per apprendere nuove tecnologie.
Sapevo, inoltre, che sarei stato inserito in un ambiente di lavoro giovane, dinamico e professionale dove la collaborazione viene ritenuta uno dei pilastri fondamentali
per la buona riuscita di ogni progetto.

Gli obiettivi che ambivo a raggiungere a livello personale riguardavano principalmente la mia formazione riguardo tecnologie a me estranee. 
In particolare, ambivo a poter apprendere il funzionamento di un sistema RFID, andando poi ad applicare quanto imparato nello sviluppo di prodotto 
capace di essere resiliente ai guasti e che si interfacciasse con un \emph{device} fisico.
In aggiunta a questo, era mia intenzione osservare ed apprendere il più possibile le metodologie di lavoro utilizzate e le \emph{best practicies}, consolidate
e utilizzate da anni dal gruppo di \emph{tech lead} presente all'interno dell'azienda. Questo per poter conoscere le strategie che, partendo da un'idea,
portano alla concretizzazione di un prodotto finito e funzionante.
%**************************************************************
\section{Vincoli}
\subsection{Vincoli tecnologici}
\label{sub-sec:vinc-tec}
Essendo la soluzione basata sulla piattaforma AMP e comprendente delle librerie native proprie del lettore, i vincoli tecnologici hanno fatto si che venissero utilizzate specifiche tecnologie:
\begin{itemize}
    \item \textbf{Java 8}: linguaggio di programmazione utilizzato per lo sviluppo;
    \item \textbf{OSGi Knopflerfish}: \emph{framework} utilizzato per la realizzazione di AMP, mette a disposizione la possibilità di creare moduli autonomi;
    \item \textbf{JNA}: libreria per accedere a codice nativo.
\end{itemize}

\subsection{Vincoli temporali}
%Il tempo determinato dallo stage, ovvero 320 ore, è stato il vincolo temporale più rilevante, in quanto è stato impiegato in larga parte per la formazione personale rispetto alle tecnolgie da utilizzare
%e alla teoria riguardante la tecnolgia RFID, indispensabile per capire il funzionamento dei lettori.
%Un altro aspetto indispensabile da considerare è stato riguardo la documentazione offerta dal produttore del lettore, questa a fine commerciali è stata considerevolmente scarsa ed ostica.
%Questo, unito alla presenza di alcuni bug o incompletezze che ho riscontrato all'interno del loro codice ha portato ad una frequente comunicazione fra me ed il loro reparto di sviluppo.
%Questo necessario scambio di informazioni ha invevitabilmente ridotto il tempo a mia disposizione per soddisfare le richieste fattemi.

Il tempo determinato dallo stage, ovvero 320 ore, è stato il vincolo temporale più rilevante, in quanto veniva incluso in esso la mia
formazione autonoma riguardo: la teoria sulla tecnologia RFID, le tecnolgie utilizzate e le librerie offerte dal produttore dei lettori.
In aggiunta a questo, le revisioni di progetto settimanali, di cui si parla in maniera approfondita nella sezione \ref{sub-sect:revisioni-progetto},
richiedono un puntuale sviluppo di ogni incremento in modo da facilitare l'organizzazione di tali incontri.
%**************************************************************
\section{Futuro del progetto}
In connessione a quanto riportato nella sezione \ref{sec:descrizione}, è facile vedere come il prodotto sviluppato fosse interpretato come uno strumento utilizzabile
in futuri progetti cliente. La visione complessiva dell'azienda, rispetto al prodotto, risultava però essere più ampia. 
L'applicativo non avrebbe dovuto essere fine a se stesso, ma rappresentare una base di studio per qualsiasi altro programmatore intento nello sviluppo 
di una soluzione basata sui lettori Kathrein o, più in generale, necessitante di interfacciarsi con del codice nativo.

%Alla fine dello stage, il prodotto sviluppato è stato revisionato dal mio \emph{tutor} interno che ha vagliato il lavoro realizzato. Tutto il codice scritto è stato inoltre corredato da opportuna
%documentazione ed è contenuto all'interno della \emph{repository} aziendale. Questo sarà utilizzabile nella definizione di nuovi flussi di lavoro o nell'aggiornamento di quelli già esistenti.
%Oltre a ciò, il lavoro svolto fornisce una base di partenza per tutti gli sviluppatori che in futuro andranno a doversi interfacciare con codice nativo, siano questi per aggiungere delle funzionalità
%al modulo o per crearne di nuovi.