% !TEX encoding = UTF-8
% !TEX TS-program = pdflatex
% !TEX root = ../tesi.tex

%**************************************************************
\chapter{Il contesto aziendale}
\label{cap:introduzione}
%**************************************************************

\intro{Nel corso di questo capitolo presento l'azienda in cui è stato svolto il progetto di stage, 
ne illustro i processi interni e la sua posizione sul mercato. 
Discuto, inoltre, le tecnologie utilizzate e la propensione aziendale nei confronti dell'innovazione.}

%Introduzione al contesto applicativo.\\

%\noindent Esempio di utilizzo di un termine nel glossario \\
%\gls{api}. \\

%\noindent Esempio di citazione in linea \\
%\cite{site:agile-manifesto}. \\

%\noindent Esempio di citazione nel pie' di pagina \\
%citazione\footcite{womak:lean-thinking} \\

%**************************************************************
\section{Storia e ambiti di interesse}

L'azienda ospitante prende il nome di Aton S.P.A., società del Trevigiano di importante rilevanza per le sue relazioni commerciali intraprese con
aziende leader italiane e numerose multinazionali.
Operante in numerosi ambiti, Aton fornisce soluzioni e servizi IT, in linea con i principi Industry 4.0, per le vendite multicanale, le catene di negozi, 
la grande distribuzione e la gestione degli \emph{asset} nei settori CPG (\emph{Consumer Packaged Goods}), Retail, Fashion ed Energy.
Proprio nel contesto di gestione degli \emph{asset} vengono introdotte due delle applicazioni tecnologiche di punta di Aton, le applicazione \gls{iotg} e \gls{m2mg}.

%\begin{figure}[!h] 
%    \centering 
%    \includegraphics[width=0.5\columnwidth]{azienda/logo.png} 
%    \caption{Logo Aton}
%\end{figure}

Data la grande varietà di ambiti in cui l'azienda è impiegata, anche il team che la compone presenta un cospicuo numero di professionisti, all'incirca 150. 
Aggiungendo a questi anche i dipendenti operanti in una rete di società partecipate, si raggiungono circa le 200 persone. 
Queste società, specializzate per area geografica e per soluzioni verticali, sono: Blue Mobility e Aton White in Italia,
mentre, per Spagna e Portogallo si trova Aton Allspark Iberica.
Non è possibile infine non citare il proficuo clima lavorativo presente all'interno dell'azienda, che in aggiunta allo spirito di innovazione e alle numerose possibilità di crescita personale
offerta dall'azienda stessa, ha permesso ad Aton di ricevere per il suo terzo anno consecutivo la certificazione "Great Place To Work", che la colloca di diritto
nella classifica delle 50 migliori aziende per cui lavorare in Italia.

%**************************************************************


%**************************************************************
\section{Organizzazione dei processi aziendali}

L'azienda al fine di gestire al meglio i propri processi interni, segue rigidamente un elenco ben definito di processi. All'interno del reparto RFID questa
pratica viene ritenuta di primaria importanza. Questo è dovuto dalla natura stessa del reparto, lavorando su diversi progetti cliente ognuno differente 
dall'altro, una struttura solida con cui ogni sviluppatore possa orientarsi è l'unica possibilità per distribuire prodotti affidabili e che soddisfino
tutte le aspettative del cliente. I processi cardine citati comprendono:
\begin{itemize}
    \item \textbf{Pianificazione e progettazione}: il reparto si occupa della creazione di flussi di lavoro coinvolgenti la tecnologia RFID sulla base delle esigenze
    specifiche espresse dal cliente, servendosi di un commerciale e di un tecnico specializzato. Una volta effettuato questo incontro viene creata un'offerta
    commerciale che viene inviata al cliente in modo che questo possa accettarla o meno. In caso di approvazione si inizia a definire in dettaglio
    i requisiti tecnici per lo sviluppo del \emph{software} e sui dispositi \emph{hardware} che dovrebbero essere utilizzati se già in possesso del cliente.
    In seguito si definiscono i dettagli del flusso di lavoro, questo è definito come l'insieme di tutti gli \emph{step} che un \emph{tag RFID} 
    effettuerà e che eventi questi debbano scatenare. Una volta che il flusso è definito si passa all'implementazione, partendo da una piattaforma proprietaria dell'azienda 
    vengono realizzate tutte le personalizzazioni necessarie in un determinato linguaggio di programmazione.
    Concluso ciò si passa alla fase di \emph{test} e collaudo in cui si verifica che l’applicazione realizzata rispetti tutte le specifiche 
    di progetto, così come collaudi funzionali e di performance.
    Infine il prodotto \emph{software} che ha superato il collaudo viene installato all'interno dell'infrastruttura di rete del cliente e grazie ad interventi
    di tecnici specializzati la piattaforma viene resa utilizzabile dagli utenti.
    \item \textbf{Assistenza e manutenzione \emph{software}}: il reparto fornisce un servizio di assistenza e supporto al cliente nell'utilizzo e nella manutenzione 
    di tutte le soluzioni sviluppate internamente, o da aziende terze con cui sono stati stretti appositi accordi commerciali. Il cliente nello specifico può
    inoltrare una richiesta di assistenza tramite l'appertura di un così detto \emph{ticket} o tramite chiamata al centralino aziendale apposito.
    Il \emph{ticket} verrà quindi preso in carico da un apposito tecnico. Una volta svolto il lavoro necessaro viene poi inviata una notifica di completamento al cliente. 
    \item \textbf{Gestione e monitoraggio infrastuttura installata}: il reparto fornice un servizio di monitoraggio sull'infrastruttura del cliente in cui il
    \emph{software} sviluppato viene installato. Questo comprende diagnostiche sullo stato della rete, sulle performance dei dispositivi \emph{hardware}
    utilizzati e di tutte le possibili anomalie causate dall'errore di un operatore umano. Grazie a questo è possibile fornire \emph{feedback}
    tempestivi al cliente in modo da poter intervenire sulla causa del problema, andando ad arginare le perdite economiche dovute ad un arresto della produzione.
\end{itemize} 

A fare da collante ai processi descritti è l'utilizzo di un \emph{way of working} orientato al modello Agile. 
Questo permette al \emph{Project Manager} di suddividere il lavoro complessivo in molti piccoli incrementi, 
ognuno di questi formato da diverse fasi, queste sono illustrate in figura \ref{agile_example}. 
Questo approccio garantisce che il processo di sviluppo sia facilmente controllabile e misurabile, ciò permette al responsabile di progetto
di rispondere in maniera efficace ad imprevisti e modifiche identificate a sviluppo avviato. 
Olte a ciò, viene data la possibilità al cliente di seguire l'avanzamento dello sviluppo del progetto, incremento dopo incremento, potendo così 
dare \emph{feedback} su quanto si è già prodotto e su cosa si andrà ad implementare. \\
Al fine di raggiungere questi obiettivi è di vitale importanza la comunicazione, ritenuta attività fondamentale della
gestione di progetto. Il gruppo di lavoro tende ad avere più contatti possibili tra gli stessi membri, 
sia durante lo \emph{smartworking} che in ufficio, così da rimanere aggiornati in modo costante riguardo il lavoro svolto. \\
Ad inizio di ogni giornata inoltre si effettua una breve riunione riguardante il progresso dei prodotti in sviluppo 
e si espongono eventuali difficoltà riscontrate nella giornata precedente.
La medesima cosa viene fatta col cliente: una volta a settimana viene fatta una \emph{call} collettiva tra tutti i partecipanti al
progetto ed i referenti dell'azienda committente, in cui vengono illustrati tutti i progressi effettuati.

\begin{figure}[!ht] 
    \centering 
    \includegraphics[width=0.8\columnwidth]{azienda/agile.png} 
    \caption{Illustrazione del modello Agile - \textbf{fonte} \url{urly.it/3g72a}}
    \label{agile_example}
\end{figure}
%**************************************************************

%**************************************************************
\section{Tecnologie utilizzate}

Come è facile intuire dai numerosi ambiti in cui l'azienda offre soluzioni e servizi, anche il numero di tecnologie utilizzate risulta essere consistente
ed in continua evoluzione. Di seguito verrano riportate le tecnologie principali utilizzate all'interno del reparto \gls{rfidg} con cui ho potuto venire in contatto
durante il periodo di stage.

\subsection{Sistemi operativi}

Per quanto concerne i sistemi operativi, vi è una distinzione nel loro utilizzo in base al ruolo che il dipendente ricopre all'interno del reparto. 
Per gli sviluppatori viene adoperato Windows, in quanto, grazie alla sua robustezza e alla facile amministrazione remota permette la creazione di ambienti
di lavoro omogenei per l'intero team. Per quanto riguarda invece la parte più amministrativa, la soluzione adottata è stata quella di dispositivi Apple
e quindi aventi MacOS. Per ultimo ma non per importanza troviamo Linux, usato come piattaforma per ospitare le soluzioni sviluppate da Aton presso le sedi
dei loro clienti.

\subsection{Ambienti di sviluppo}

Per la scrittura del codice non viene espressa una preferenza specifica, agli sviluppatori viene lasciata la possibilità di utilizzare una delle seguenti
\emph{IDE}: Eclipse, Intelij Idea o Visual Studio. La scelta di una di queste ricade in primo luogo sulle esigenze specifiche del progetto su cui lo
sviluppatore si trova a lavorare, ed in secondo luogo sulla propria preferenza personale. Ognuno di questi programmi porta con se un numero molto elevanto
di funzionalità, queste se combinate con ambienti di lavoro pre-configurati allestiti negli anni e disponibili agli sviluppatori, vanno a formare delle
soluzioni pronte che rendono lo spostamento da un progetto ad un altro semplice ed immediato.
Oltre ai prodotti già citati, non è possibile non citare la rapida diffusione negli ultimi anni di un altro strumento dalla grande popolarità, 
l'\emph{editor} Visual Studio Code. 
Questo programma multipiattaforma nasce con lo scopo di essere un \emph{editor} "tutto fare",
di base infatti è provvisto solo di strumenti avanzati per la scrittura e la formattazione del testo. Grazie però ad un vasto e ricco market di estensioni,
è possibile una sua profonda e completa personalizzazione integrando quindi il supporto a tutti i linguaggi di programmazione più rilevanti presenti sul
mercato. Questo porta alla creazione di un ambiente dinamico che possa rispondere a tutte le esigenze di ogni sviluppatore.

\subsection{Sistemi di versionamento}

Dato il numeroso team di lavoro risulta indispensabile fare affidamento su un sistema di versionamento, questo tiene traccia del flusso 
di lavoro e di tutte le modifiche che vengono apportate al codice sorgente. Storicamente, all'interno dell'azienda, il punto di riferimento per questo
compito è stato il \emph{software open source} Subversion. Questo però sta lentamente venendo abbandonato per effettuare una massiccia migrazione
verso Git, altro \emph{software} per il controllo di versione del codice distribuito che è diventanto di fatto uno standard per il mercato odierno.

\subsection{Suite per la produttivita e comunicazione}
Per quanto riguarda l'organizzazione e la comunicazione viene fatto un ampio uso dei servizi proposti da Microsoft. Di questi, vengono usati principalmente:
\begin{itemize}
    \item \textbf{Outlook}: servizio di posta elettronica che comprende anche un \emph{client} desktop ed un calendario condivisibile. 
    Viene utilizzato sia internamente tra i dipendenti che coi clienti.
    \item \textbf{Onedrive}: servizio \emph{web} per la memorizzazione, sincronizzazione e condivisione di file online.
    \item \textbf{Office Suite}: pacchetto di applicazioni \emph{software} che presenta numerose funzionalità, come la creazione e modifica di \emph{file} di testo, fogli di calcolo, presentazioni e reportistica varia.
    \item \textbf{Teams}: piattaforma di comunicazione e collaborazione unificata che combina chat di lavoro persistente, teleconferenza, condivisione di contenuti e integrazione delle applicazioni esterne.
\end{itemize}

\subsection{Linguaggi di sviluppo}

Avendo l'esigenza di sviluppare progetti e soluzioni \emph{software} che si interfaccino frequentemente con dispositivi \emph{hardware} e che possano operare andando ad interfacciarsi
con un importante numero di applicativi scritti da terzi, viene fatto uso di diversi linguaggi di programmazione:

\begin{itemize}
    \item \textbf{Java}: basato su Java Virtual Machine(JVM), ormai da svariati anni è il più utilizzato per il suo vastissimo numero di librerie. Questo,
    combinato alla possibilità di eseguire il codice in modo parallelo e concorrente
    lo rende uno strumento solido ed affidabile. Un altro vantaggio di tale linguaggio riguarda l'esecuzione del codice stesso, questo infatti risulta indipendente dall’\emph{hardware} della macchina in quanto virtualizzato 
    dalla piattaforma stessa. In tal modo il codice è reso portabile verso altre piattaforme con \emph{hardware} diversi tra loro.
    \item \textbf{C\#}: linguaggio di programmazione multiparadigma, viene sviluppato da Microsoft e presenta numerose somiglianze con Java. Negli ultimi anni ha avuto una forte crescita dovuta ad alcune sue caratteristiche uniche
    come la possibilità di interfacciare nativamente codice scritto in linguaggi compilati come C e C++. Questo, unito ad un solido ecosistema di librerie e dalle sue notevoli performance, ha spinto numerosi produttori ad 
    adottarlo come strumento per interfacciarsi con i propri dispositivi \emph{hardware}.
    \item \textbf{Javascript}: linguaggio di programmazione orientato agli oggetti tramite prototipi e agli eventi. Grazie alla sua grande popolarità ed alla
    produttività nella scrittura del codice viene utilizzato per prototipizzazione o per specifici ambiti.
\end{itemize}

\subsection{Framework}

Il principale e sicuramente più utilizzato è OSGi Knopflerfish, un \emph{framework open-source} che si propone di implementare 
un modello a componenti completo e dinamico per la piattaforma Java. I concetti fondamentali dietro questa tecnologia sono essenzialmente 3:
\begin{itemize}
    \item Definizione del concetto di modulo (\emph{bundle})
    \item Gestione automatica delle dipendenze
    \item Gestione del ciclo di vita del codice (configurazione e distribuzione dinamica)
\end{itemize}
Grazie a ciò viene reso possibile creare un sistema dinamico, che consente l'installazione, l'avvio, lo stop e la rimozione dei moduli a \emph{runtime}, 
senza quindi necessitare di riavvii. Un esempio di come il codice venga organizzato tramite il concetto di modulo è visibile nella figura \ref{osgi_example}. 
Non a caso quindi è stato adottato per la creazione della principale piattaforma utilizzata dal reparto RFID, AMP, 
che offrendo la possibilità di creare nuovi moduli si rende una perfetta base di partenza per progetti orientati alla personalizzazione per il singolo cliente.

\begin{figure}[!ht] 
    \centering 
    \includegraphics[width=0.8\columnwidth]{azienda/osgi.png} 
    \caption{Architettura con OSGi - \textbf{fonte} \url{urly.it/3g72y}}
    \label{osgi_example}
\end{figure}
%**************************************************************

%**************************************************************
\section{Propensione all'innovazione}

L'azienda da molti anni ha deciso di mettere la formazione continua come uno dei suoi pilastri principali, operando in mercati dinamici e con tecnologie in
continua evoluzione risulta quindi indispensabile continuare ad ampliare le proprie conoscenze, in modo da non essere superati dai vari \emph{competitor}.
A tal fine vengono messi a disposizione di ogni dipendente numerosi corsi di formazione, questi spaziano da certificazioni per tecnologie specifiche, corsi
sulla gestione di progetto fino ad arrivare a corsi di lingue straniere. \\
Il reparto RFID non fa eccezione, viene richiesto infatti a tutti gli sviluppatori di seguire percorsi formativi su un ampia scelta di tecnolgie,
identificate come cruciali per il continuo rinnovamento ed espansione dell'azienda. Questi comprendono ad esempio corsi su tecnoglie riguardanti la persistenza
dei dati fine all'utilizzo di \emph{framework} di nuova generazione.
Oltre a questo Aton ha dimostrato negli anni una forte connessione con l'ambiente universitario e quindi di conseguenza con la ricerca che si svolge in esso.
Uno degli esempi più significati è sicuramente la collaborazione avvenuta con l'Università di Torino con cui si sono svolte delle ricerche riguardanti proprio
la tecnologia RFID. Queste hanno portato alla creazione delle basi della piattaforma proprietaria per tale tecnologia, AMP, utilizzata ancora oggi come base di
partenza per tutti i progetti cliente sviluppati. \\
Correlata alla connessione con gli ambienti universitari appena citati, troviamo l'occupazione di un dipendente Aton, che oltre al suo lavoro in azienda svolge
l'attività di professore nel corso "Altri paradigmi di programmazione" all'interno della laurea in Informatica presso l'Università di Padova. 
%**************************************************************


%\section{Organizzazione del testo}

%Riguardo la stesura del testo, relativamente al documento sono state adottate le seguenti convenzioni tipografiche:
%\begin{itemize}
%	\item gli acronimi, le abbreviazioni e i termini ambigui o di uso non comune menzionati vengono definiti nel glossario, situato alla fine del presente documento;
%	\item per la prima occorrenza dei termini riportati nel glossario viene utilizzata la seguente nomenclatura: \emph{parola}\glsfirstoccur;
%	\item i termini in lingua straniera o facenti parti del gergo tecnico sono evidenziati con il carattere \emph{corsivo}.
%\end{itemize}