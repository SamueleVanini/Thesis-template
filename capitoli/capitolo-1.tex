% !TEX encoding = UTF-8
% !TEX TS-program = pdflatex
% !TEX root = ../tesi.tex

%**************************************************************
\chapter{Il contesto aziendale}
\label{cap:introduzione}
%**************************************************************

\intro{Nel corso di questo capitolo si presenterà l'azienda in cui è stato svolto il progetto di stage. Si andranno ad illuistare i processi interni di questa
e la sua posizione sul mercato. Oltre a questo verranno illustrate le tecnologie utilizzatw e la sua propensione nei confronti dell'innovazione}\\

%Introduzione al contesto applicativo.\\

%\noindent Esempio di utilizzo di un termine nel glossario \\
%\gls{api}. \\

%\noindent Esempio di citazione in linea \\
%\cite{site:agile-manifesto}. \\

%\noindent Esempio di citazione nel pie' di pagina \\
%citazione\footcite{womak:lean-thinking} \\

%**************************************************************
\section{Storia e ambiti di interesse}

L'azienda ospitante prende il nome di Aton S.P.A., società del Trevigiano di importante rilevanza per le sue relazioni commerciali intraprese con\
aziende leader italiane e numerose multinazionali.
Operante in numerosi ambiti, Aton fornisce soluzioni e servizi IT, in linea con i principi Industry 4.0, per le vendite multicanale, le catene di negozi, 
la grande distribuzione e la gestione degli asset nei settori CPG (Consumer Packaged Goods), Retail, Fashion ed Energy.
Proprio nel contesto di gestione degli asset vengono introdotte due delle applicazioni tecnologiche di punta di Aton, le applicazione \gls{iotg} e \gls{m2mg}

\begin{figure}[!h] 
    \centering 
    \includegraphics[width=0.5\columnwidth]{azienda/logo.png} 
    \caption{Logo Aton}
\end{figure}

Data la grande varietà di ambiti in cui l'azienda è impiegata, anche il team che la compone presenta un cospicuo numero di professionisti, all'incirca 150. 
Aggiungendo a questi anche i dipendenti operanti in una rete di società partecipate, si raggiungono circa le 200 persone. 
Queste società, specializzate per area geografica e per soluzioni verticali, sono: ​Blue Mobility e Aton White in Italia,
mentre per Spagna e Portogallo si trova Aton Allspark Iberica.
Risulta importante citare la grande propensione dell'azienda nei confronti dell'innovazione e di una formazione continua per tutti i suoi dipendenti.
Sono presenti, infatti, numerosi corsi di formazione personale gestiti da appositi professionisti a cui l'azienda invita a partecipare. Oltre a ciò,
sono anni che vengono intraprese relazioni con istututi accademici del territorio, questo ha portato a molto opportunità di stage alli'interno dell'azienda
e ad una collaborazione con il politecnico di Torino circa una decina di anni fa, per lo studio e la realizzazione di una piattaforma per dispositivi RFID
durante la nascita di questa tecnologia in Italia. Occorre infine citare la presenza di lavoratore di Aton, che svolge attività come professore del corso
"Altri paradigmi di programmazione" nel corso di informatica all'Università di Padova.
Grazie al clima lavorativo proficuo instauratosi all'interno dell'azienda, allo spirito di innovazione e alle numerose possibilit di crescita personale
offerta dall'azienda stessa, Aton ha ricevuto per il suo terzo anno consecutivo la certificazione "Great Place To Work", che la colloca di diritto
nella classifica delle 50 migliori aziende per cui lavorare in Italia.

%**************************************************************

%**************************************************************
\section{Tecnologie utilizzate}

Spiegazioni delle tecnologie adoperate (capire se da tutta l'azienda o solo nel tirocinio)

%**************************************************************

%**************************************************************
\section{Organizzazione dei processi aziendali}

Spiegazioni dei processi all'interno di Aton

%**************************************************************

%**************************************************************
\section{Propensione all'innovazione}

%**************************************************************



%\section{Organizzazione del testo}

%Riguardo la stesura del testo, relativamente al documento sono state adottate le seguenti convenzioni tipografiche:
%\begin{itemize}
%	\item gli acronimi, le abbreviazioni e i termini ambigui o di uso non comune menzionati vengono definiti nel glossario, situato alla fine del presente documento;
%	\item per la prima occorrenza dei termini riportati nel glossario viene utilizzata la seguente nomenclatura: \emph{parola}\glsfirstoccur;
%	\item i termini in lingua straniera o facenti parti del gergo tecnico sono evidenziati con il carattere \emph{corsivo}.
%\end{itemize}