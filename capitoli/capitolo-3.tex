% !TEX encoding = UTF-8
% !TEX TS-program = pdflatex
% !TEX root = ../tesi.tex

%**************************************************************
\chapter{Resoconto del progetto}
\label{cap:resoconto-progetto}
%**************************************************************

\intro{Nel corso di questo capitolo si presenta un resoconto del progetto, descrivendone gli aspetti nel dettaglio. Si illustrano le metodologie di lavoro utilizzato e le scelte
dietro lo sviluppo delle componenti ritenute più rilevanti o critiche}

%**************************************************************
\section{Pianificazione delle attività}
In accordo con il mio \emph{tutor} interno abbbiamo stilato un piano di lavoro suddividendo gli argomenti da affrontare di settimana in settimana, indicando poi per ciascuna settimana
le ore di lavoro e le attività che avrei svolto. \\
Viene riportato di seguito il piano di lavoro in forma tabellare:

\begin{table}
\label{tab:piano-di-lavoro}
\begin{tabularx}{\textwidth}{ | X | X | X |}
\hline
\textbf{Ore settimanali} & \textbf{Attività svolte}\\
\hline
Prima settimana - 40 ore & Presentazione del progetto e degli attori coinvolti, predisposizione dell’ambiente di lavoro, 
formazione sugli apparati \emph{hardware} adottati e inizio studio autono sulle tecnologie adottate. \\
\hline
Seconda settimana - 40 ore & Studio autonomo su: tecnolgie e \emph{framework} adottati, teoria RFID, librerie offerte dal produttore degli apparati utilizzati \\
\hline
Terza settimana - 40 ore & Progettazione del modulo software ed implementazione delle componenti riguardanti letture tag e invio dati alla piattaforma AMP\\
\hline
Quarta settimana - 40 ore & Implementazione delle componenti riguardanti \emph{sessions} e \emph{singulation}\\
\hline
Quinta settimana - 40 ore & Implementazione delle componenti riguardanti \emph{transit time} e \emph{dwell time}\\
\hline
Sesta settimana - 40 ore & Implementanzione delle componenti riguardanti \emph{kray protocol} e gestione porte \emph{GPIO} \\
\hline
Settima settimana - 40 ore & Implementazione della componente riguardante il controllo del modulo dalla piattaforma AMP\\
\hline
Ottava settimana - 40 ore & Test approfonditi del prodotto, stesura della documentazione relativa a quanto implementato, 
stesura di un documento riassuntivo sulla tecnologia JNA e sulle librerie offerte dal produttore.\\
\hline
\end{tabularx}
\caption{Piano di lavoro}
\end{table}

\subsection{Comunicazioni}
Durante tutta la durata le progetto ho tenuto una comunicazione attiva con tutte le figure aziendali interessate al prodotto da me sviluppato. \\
Ho tenuto giornalmente delle riunioni con il mio \emph{tutor} interno, della durata di circa 15 minuti in cui esponevo lo stato di avanzamento, riportando
quanto svolgo durante nell'ultima giornata trascorsa, i problemi sorti e le mie proposte su come su come sviluppare le componenti richieste.
Ho partecipato regolarmente alle riunioni di reparto svolte ogni lunedì della durata di circa un'ora in ho avuto la possibilità di rendere participi tutti
gli sviluppatori presenti del lavoro da me svolto e avessi adoperato tecnologie o tecniche ritenute da me di interesse collettivo.
Oltre agli incontri precedentemente citati, ho tenuto periodicamente delle revisioni di codice con uno dei dipendenti appartenente al gruppo di \emph{tech lead}
per discutere delle scelte progettuali ed implementative da me fatte, questo verrà approfondito nel capitolo che segue.

\subsection{Revisioni di progetto}
\label{sub-sect:revisioni-progetto}
La progettazione e lo sviluppo del progetto sono state intervallate da periodici controlli sulla qualità di quanto da me prodotto. Questi controlli erano
effettuati insieme ad un sviluppatore esperto facente parte del gruppo di \emph{tech lead} presente in azienda. Durante questi incrontri tutto il codice sorgente
da me scritto veniva revisionato e ne venivano commentate le caratteristiche e le motivazioni dietro ogni mia scelta. Al termine di questa fase, il prodotto
veniva testato andando a creare un esempio di una linea di produzione in cui potevo effettuare test di carico o osservare i comportamenti del prodotto a
situazioni comuni a cui sottoporsi.

%**************************************************************
\section{Analisi del progetto}
Nonostante abbia sviluppato l'intero applicativo in autonomia, ho seguito tutti i passi adoperati dal gruppo di sviluppo aziendale per la realizzazione
di un progetto interno e quindi non finalizzato ad un cliente specifico.
Questi sono elencati di seguito:
\begin{itemize}
    \item \textbf{Incontro con gli \emph{stakeholder}}: in questa fase viene fatto l’incontro tra il gruppo che andrà a sviluppare il prodotto e tutti 
    gli attori interassati al proggetto. In questa sede vengono discusse varie idea, specificando i requisiti e gli obiettivi che il prodotto dovrà 
    raggiungere;
    \item \textbf{\emph{Scouting} delle tecnologie}: dopo aver acquisito le informazioni utili relative al progetto, il gruppo di lavoro inizia a ricercare 
    le tecnologie che possono risultare più adatte a soddisfare le richieste poste dal committente vagliando anche quelle più sperimentali ed innovative;
    \item \textbf{Studio di fattibilità}: successivamente viene fatto lo studio di fattibilità, in modo tale da poter capire se il prodotto si può sviluppare 
    concretamente. Viene svolta anche un’analisi dei rischi e dei costi, al fine di poter presentare a tutti gli interessati al progetto una chiara panoramica
    di quello che ci si aspetta essere il risultato finale. In questa fase il progetto può essere scartato se considerato non abbastanza vantaggioso;
    \item \textbf{Sviluppo}: se gli attori coinvolti accettano quanto riportato dallo studio di fattibilità, allora si inizia con la fase vera 
    e propria dello sviluppo del prodotto. 
    Il lavoro viene suddiviso in tanti incrementi, come descritto nel modello Agile, così da averlo sempre sotto controllo e poter sempre mostrare 
    al committente i risultati che si stanno ottenendo;
    \item \textbf{\emph{Testing}}: in questa fase il prodotto viene testato mediante specifici test, sia manuali che automatici, per verificarne le 
    funzionalità e assicurarsi che il prodotto abbia soddisfatto le richieste del committente.
\end{itemize}

\subsection{Funzionalità principali}
Dopo aver impostato gli incrementi, in accordo con il mio \emph{tutor} interno, ho redatto una lista di macro funzionalità che il prodotto avrebbe 
dovuto avere:
\begin{itemize}
    \item \textbf{Funzionamento autonomo}: dovendo integrare questi lettori in contesti industriali in cui il carico di lavoro risulta essere particolarmente 
    elevato, era necessario che i dispositivi potessero lavorare senza la presenza di un operatore umano, che gli azionasse e controllasse;
    \item \textbf{Tolleranza ai guasti}: operando in contesti in cui agenti fisici possono facilmente interrompere la connessione ai dispositivi, era necesserio
    identificare e gestire tali anomalie;
    \item \textbf{Configurazione a "caldo"}: necessitando di cambi di configurazione durante il suo utilizzo, il prodotto doveva poter modificare alcuni 
    parametri di configurazione senza la necessità di riavviarsi o di interrompere il flusso di lavoro in cui inserito;
    \item \textbf{Integrazione con AMP}: dovendo inviare dati alla  piattaforma utilizzata per la realizzazione dei flussi di lavoro e ricevere comandi da essa, 
    era necessario implementare una strategia di comunicazione con tale \emph{software};
\end{itemize}
\subsection{Obiettivi e requisiti chiave}

\subsubsection*{Obiettivi}
Durante la fase di stesura del Piano di Lavoro, io ed il mio \emph{tutor} interno abbiamo prefissato degli obiettivi da raggiungere entro la fine dello stage.
Di seguito sono riportati mediante una nomenclatura che li identifica univocamente:
\begin{center}
    [\textbf{classificazione}][\textbf{numero incrementale}]
\end{center}
dove il campo classificazione può avere uno dei seguenti valori:
\begin{itemize}
    \item \textbf{O}: requisiti obbligatori, vincolanti in quanto obiettivo primario richiesto dal committente;
    \item \textbf{D}: requisiti desiderabili, non vincolanti o strettamente necessari, ma dal riconoscibile valore aggiunto;
    \item \textbf{F}: requisiti facoltativi, rappresentanti valore aggiunto non strettamente competitivo.
\end{itemize}
Vengono di seguito riportati in forma tabellare.
\begin{table}
    \label{tab:obiettivi}
    \begin{tabularx}{\textwidth}{ | X | X |}
    \hline
    \textbf{Obiettivi} & \textbf{Descrizione}\\
    \hline
    O01 & Dimostrazione di una piena conoscenza della teoria legata alla tecnologia RFID \\
    \hline
    O02 & Realizzazione di un modulo software pienamente funzionante che rispetti le caratteristiche desiderate dagli stakeholder \\
    \hline
    O03 & Realizzazione della documentazione su quanto implementato e sulle tecnologie utilizzate \\
    \hline
    D01 & Realizzazione di uno studio di fattibilità sul \emph{porting} del modulo realizzato all'interno dei lettori RFID \\
    \hline
    F01 & Esposizione tramite API dello stato fisico dei lettori \\
    \hline
    \end{tabularx}
    \caption{Piano di lavoro}
\end{table}

\subsubsection*{Analisi dei requisiti}
Al fine di raggiungere gli obiettivi elencati, io ed il mio \emph{tutor} interno, ci siamo concentrati nella raccorta di requisiti. Questi sono identificati
da una nomenclatura univoca, cosí composta:
\begin{center}
    \textbf{R}[\textbf{classificazione}][\textbf{tipologia}][\textbf{numero incrementale}]
\end{center}

Per \textbf{classificazione} del requisito si intende:
\begin{itemize}
    \item \textbf{O}: obbligatoro;
    \item \textbf{D}: desiderabile;
    \item \textbf{F}: facoltativo.
\end{itemize}

Per \textbf{tipologia} del requisito si intende:
\begin{itemize}
    \item \textbf{F}: funzionale;
    \item \textbf{Q}: qualitativo;
    \item \textbf{P}: prestazionale;
    \item \textbf{V}: vincolo.
\end{itemize}

Di seguito sono riportati solo i requisiti funzionali estratti dai Casi d'Uso più significativi presentati in sezione \ref{sub-sec:use-case}.
\begin{table}
    \label{tab:requisiti-fun}
    \begin{tabularx}{\textwidth}{ | X | X | X |}
    \hline
    \textbf{Requisito} & \textbf{Descrizione} & \textbf{Fonte}\\
    \hline
    ROF1 & L'utente deve poter visionare la configurazione del lettore & UC1 \\
    \hline
    ROF2 & L'utente deve poter modificare la configurazione del lettore & UC2 \\
    \hline
    ROF3 & L'applicativo deve poter tornare in funzione in maniera autonoma a seguito di errori dovuti ala rete & UC5 \\
    \hline
    ROF4 & L'utente deve poter modificare lo stato delle porte GPIO connesse al lettore & UC3 \\
    \hline
    ROF5 & Il modulo deve inviare i dati relativi ai tag letti alla piattaforma AMP & UC4 \\
    \hline
    \end{tabularx}
    \caption{Tabella di tracciamento dei requisiti funzionali}
\end{table}

Per quanto riguarda i requisiti di vincolo:
\begin{table}
    \label{tab:requisiti-vinc}
    \begin{tabularx}{\textwidth}{ | X | X | X |}
    \hline
    \textbf{Requisito} & \textbf{Descrizione} & \textbf{Fonte}\\
    \hline
    ROV1 & Utilizzo di Java 8 per la codifica del modulo & Interna \\
    \hline
    ROV2 & Utilizzo della libreria JNA per interfacciarsi con librerie native & Interna \\
    \hline
    \end{tabularx}
    \caption{Tabella di tracciamento dei requisiti di vincolo}
\end{table}

I requisiti qualitativi rilevati sono i seguenti:
\begin{table}
    \label{tab:requisiti-qual}
    \begin{tabularx}{\textwidth}{ | X | X | X |}
    \hline
    \textbf{Requisito} & \textbf{Descrizione} & \textbf{Fonte}\\
    \hline
    ROQ1 & Il codice sorgente dev'essere versionato tramite il \emph{software} Git & Interna \\
    \hline
    ROQ2 & Dev'essere realizzato un documento riguardante le tecnologie, le scelte implementative e progettuali con relative motivazioni & Interna \\
    \hline
    \end{tabularx}
    \caption{Tabella di tracciamento dei requisiti qualitativi}
\end{table}

I requisiti prestazionali rilevati sono i seguenti:
\begin{table}
    \label{tab:requisiti-pre}
    \begin{tabularx}{\textwidth}{ | X | X | X |}
    \hline
    \textbf{Requisito} & \textbf{Descrizione} & \textbf{Fonte}\\
    \hline
    ROP1 & Il modulo deve supportare la lettura di centinaia di tag simultaneamente & Interna \\
    \hline
    \end{tabularx}
    \caption{Tabella di tracciamento dei requisiti qualitativi}
\end{table}


\subsection{Casi d'uso}
\label{sub-sec:use-case}

\begin{usecase}{1}{Visualizzazione configurazione del lettore}
\usecaseactors{Utente}
\usecasepre{L'utente è entrato all'interno di AMP ed ha lanciato il comando per visualizzare la configurazione del lettore}
\usecasedesc{AMP mette a disposizione una finestra relativa al modulo in cui lanciare comandi verso questo tra cui visualizzazione e modifica 
della configurazione del lettore}
\usecasepost{Il modulo restituisce ad AMP le informazioni riguardanti la configurazione che vengono mostrate a schermo}
\label{uc:1}
\end{usecase}

\begin{figure}[!h] 
    \centering 
    \includegraphics[width=0.9\columnwidth]{usecase/UC1} 
    \caption{Use Case - UC1: Visualizzazione configurazione del lettore}
\end{figure}

\begin{usecase}{2}{Mofidica configurazione del lettore}
\usecaseactors{Utente}
\usecasepre{L'utente è entrato all'interno di AMP ed ha lanciato il comando per modificare la configurazione del lettore}
\usecasedesc{AMP mette a disposizione una finestra relativa al modulo in cui lanciare comandi verso questo tra cui visualizzazione e modifica 
della configurazione del lettore}
\usecasepost{Il modulo restituisce ad AMP un feedback sull'esito dell'operazione e nel caso di esito positivo la configurazione inserita}
\label{uc:2}
\end{usecase}

\begin{figure}[!h] 
    \centering 
    \includegraphics[width=0.9\columnwidth]{usecase/UC2} 
    \caption{Use Case - UC2: Modifica configurazione del lettore}
\end{figure}

\begin{usecase}{3}{Modifica dello stato delle porte GPIO}
\usecaseactors{Utente}
\usecasepre{L'utente è entrato all'interno di AMP ed ha lanciato il comando per modificare lo stato delle porte GPIO}
\usecasedesc{AMP mette a disposizione una finestra relativa al modulo in cui lanciare il comando relativo alla modifica dello stato di una porta GPIO}
\usecasepost{Il modulo restituisce ad AMP un feedback sull'esito dell'operazione}
\label{uc:3}
\end{usecase}

\begin{figure}[!h] 
    \centering 
    \includegraphics[width=0.9\columnwidth]{usecase/UC3} 
    \caption{Use Case - UC3: Modifica dello stato delle porte GPIO}
\end{figure}

\begin{usecase}{4}{UC4: Visualizzazione configurazione del lettore}
\usecaseactors{Modulo, AMP}
\usecasepre{Il modulo ha raccolto dei dati che vuole mandare ad AMP}
\usecasedesc{AMP mette a disposizione un canale di comunicazione per l'invio di dati da un modulo qualsiasi ad essa}
\usecasepost{AMP restituisce al modulo un feeback sull'esito dell'operazione}
\label{uc:4}
\end{usecase}

\begin{figure}[!h] 
    \centering 
    \includegraphics[width=0.9\columnwidth]{usecase/UC4} 
    \caption{Use Case - UC4: Visualizzazione configurazione del lettore}
\end{figure}

\begin{usecase}{5}{UC5: Rispistino connessione al lettore in caso di anomalia}
\usecaseactors{Modulo}
\usecasepre{Il modulo riscontra una disconnessione dal lettore}
\usecasedesc{Il modulo periodicamente controlla l'esistenza di una connessione attiva e valida verso il lettore in operazione}
\usecasepost{Il modulo entra in uno stato di \emph{recovery} in cui cerca di ripristinare la connessione e notifica l'utente dell'accaduto}
\label{uc:5}
\end{usecase}

\subsection{Tecnologie coinvolte}
Per lo sviluppo del modulo, come accennato in sezione \ref{sub-sec:vinc-tec}, sono state utilizzate le seguenti tecnologie:
\begin{itemize}
    \item \textbf{OSGi Knopflerfish}: si tratta di un \emph{framework open-source} sviluppato da Makewave che rispecchia gli standard della OSGi Alliance.
    Questa fondazione, sovvenzionata a sua volta dalla Eclipse Foundation, si pone l'obiettivo di formalizzare l'evoluzione di tecnolgie industriali per
    la creazione di soluzioni modulari sulla piattaforma JVM.
    Knopflerfish offre molte \emph{feature} per agevolare lo sviluppo dell'applicativo, tra cui:
    \begin{itemize}
        \item Componenti per la creazione di un \emph{bundle} integrabile direttamente all'interno di AMP;
        \item Strumenti per l'accesso ad una \emph{thread pool} gestista dalle grandi performance;
        \item Un canale di cominicazione per lo scambio di dati tra i \emph{bundle}.
    \end{itemize}
    \item \textbf{JNA}: è una libreria \emph{open-source} che permette di accedere a codice nativo, ovvero scritto in un linguaggio compilato come C e C++, 
    direttamente dal codice scritto in Java. Il suo utilizzo è particolarmente adatto a contesti industriali dove si vuole permettere l'interazione con
    codice che possa operare direttamente sull'\emph{hardware} del dispositivo.
    Le principali \emph{feature} che offre per raggiungere questo intento sono:
    \begin{itemize}
        \item Conversione o rappresentazione di tipi di dati nativi nelle controparti utilizzate dalla JVM;
        \item Astrazione del paradigma di programmazione procedurale in programmazione ad oggetti;
        \item Gestione autonoma della memoria utilizzata dal codice nativo.  
    \end{itemize}
    \item \textbf{Java 8}: linguaggio di programmazione orientato agli oggetti, scelto per lo sviluppo dell'applicativo per la sua integrazione con 
    Knopflerfish. La scelta relativa la versione è stata presa a valle di alcune considerazioni, la \emph{release} 8 infatti presenta numerose migliorie
    come l'introduzione dei metodi lambda, conservando inoltre tutte le aggiunte alla libreria di base riguardante il parallelismo e la concorrenza introdotte
    nella versione 7. In aggiunta a questo, l'utilizzo di questa versione avrebbe impedito il versificarsi di conflitti fra JVM differenti all'interno della
    stessa soluzione
\end{itemize}

%Durante la fase di analisi iniziale sono stati individuati alcuni possibili rischi a cui si potrà andare incontro.
%Si è quindi proceduto a elaborare delle possibili soluzioni per far fronte a tali rischi.\\

%\begin{risk}{Performance del simulatore hardware}
%    \riskdescription{le performance del simulatore hardware e la comunicazione con questo potrebbero risultare lenti o non abbastanza buoni da causare il fallimento dei test}
%    \risksolution{coinvolgimento del responsabile a capo del progetto relativo il simulatore hardware}
%    \label{risk:hardware-simulator} 
%\end{risk}

%**************************************************************
\section{Sviluppo delle componenti critiche}
\subsection{Design del modulo}
\subsection{Interfacciamento con librarie native}
\subsection{Ottimizzazione dei tempi di lettura}
\subsection{Linguaggio di dominio specifico}
\subsection{Controllo delle porte GPIO}
\subsection{Tolleranza ai guasti}


%**************************************************************
\section{Risultati raggiunti}
\subsection{Requisiti soddisfatti}
\subsection{Prodotti creati}