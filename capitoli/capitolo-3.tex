% !TEX encoding = UTF-8
% !TEX TS-program = pdflatex
% !TEX root = ../tesi.tex

%**************************************************************
\chapter{Resoconto del progetto}
\label{cap:resoconto-progetto}
%**************************************************************

\intro{Breve introduzione al capitolo}\\

%**************************************************************
\section{Pianificazione delle attività}
\subsection{Comunicazioni}
\subsection{Revisioni di progetto}

%**************************************************************
\section{Analisi del progetto}
\subsection{Funzionalità}
\subsection{Requisiti}
\subsection{Casi d'uso}
\subsection{Tecnologie coinvolte}

Durante la fase di analisi iniziale sono stati individuati alcuni possibili rischi a cui si potrà andare incontro.
Si è quindi proceduto a elaborare delle possibili soluzioni per far fronte a tali rischi.\\

\begin{risk}{Performance del simulatore hardware}
    \riskdescription{le performance del simulatore hardware e la comunicazione con questo potrebbero risultare lenti o non abbastanza buoni da causare il fallimento dei test}
    \risksolution{coinvolgimento del responsabile a capo del progetto relativo il simulatore hardware}
    \label{risk:hardware-simulator} 
\end{risk}

%**************************************************************
\section{Design e sviluppo}
\subsection{Design del modulo}
\subsection{Interfaccia con librarie native}
\subsection{Perfomance delle letture}
\subsection{DSL per l'interazione con il modulo}
\subsection{Controllo delle porte GPIO}
\subsection{Fault tolerance}


%**************************************************************
\section{Risultati raggiunti}
\subsection{Requisiti soddisfatti}
\subsection{Prodotti creati}