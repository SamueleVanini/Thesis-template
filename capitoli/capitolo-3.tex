% !TEX encoding = UTF-8
% !TEX TS-program = pdflatex
% !TEX root = ../tesi.tex

%**************************************************************
\chapter{Resoconto del progetto}
\label{cap:resoconto-progetto}
%**************************************************************

\intro{Nel corso di questo capitolo si presenta un resoconto del progetto, descrivendone gli aspetti nel dettaglio. Si illustrano le metodologie di lavoro utilizzato e le scelte
dietro lo sviluppo delle componenti ritenute più rilevanti o critiche}

%**************************************************************
\section{Pianificazione delle attività}
In accordo con il mio \emph{tutor} interno abbbiamo stilato un piano di lavoro suddividendo gli argomenti da affrontare di settimana in settimana, indicando poi per ciascuna settimana
le ore di lavoro e le attività che avrei svolto. \\
Viene riportato di seguito il piano di lavoro in forma tabellare:

\begin{table}
\label{tab:piano-di-lavoro}
\begin{tabularx}{\textwidth}{ | X | X | X |}
\hline
\textbf{Ore settimanali} & \textbf{Attività svolte}\\
\hline
Prima settimana - 40 ore & Presentazione del progetto e degli attori coinvolti, predisposizione dell’ambiente di lavoro, 
formazione sugli apparati \emph{hardware} adottati e inizio studio autono sulle tecnologie adottate. \\
\hline
Seconda settimana - 40 ore & Studio autonomo su: tecnolgie e \emph{framework} adottati, teoria RFID, librerie offerte dal produttore degli apparati utilizzati \\
\hline
Terza settimana - 40 ore & Progettazione del modulo software ed implementazione delle componenti riguardanti letture tag e invio dati alla piattaforma AMP\\
\hline
Quarta settimana - 40 ore & Implementazione delle componenti riguardanti \emph{sessions} e \emph{singulation}\\
\hline
Quinta settimana - 40 ore & Implementazione delle componenti riguardanti \emph{transit time} e \emph{dwell time}\\
\hline
Sesta settimana - 40 ore & Implementanzione delle componenti riguardanti \emph{kray protocol} e gestione porte \emph{GPIO} \\
\hline
Settima settimana - 40 ore & Implementazione della componente riguardante il controllo del modulo dalla piattaforma AMP\\
\hline
Ottava settimana - 40 ore & Test approfonditi del prodotto, stesura della documentazione relativa a quanto implementato, 
stesura di un documento riassuntivo sulla tecnologia JNA e sulle librerie offerte dal produttore.\\

\hline
\end{tabularx}
\caption{Piano di lavoro}
\end{table}

\subsection{Comunicazioni}
Durante tutta la durata le progetto ho tenuto una comunicazione attiva con tutte le figure aziendali interessate al prodotto da me sviluppato. \\
Ho tenuto giornalmente delle riunioni con il mio \emph{tutor} interno, della durata di circa 15 minuti in cui esponevo lo stato di avanzamento, riportando
quanto svolgo durante nell'ultima giornata trascorsa, i problemi sorti e le mie proposte su come su come sviluppare le componenti richieste.
Ho partecipato regolarmente alle riunioni di reparto svolte ogni lunedì della durata di circa un'ora in ho avuto la possibilità di rendere participi tutti
gli sviluppatori presenti del lavoro da me svolto e avessi adoperato tecnologie o tecniche ritenute da me di interesse collettivo.
Oltre agli incontri precedentemente citati, ho tenuto periodicamente delle revisioni di codice con uno dei dipendenti appartenente al gruppo di \emph{tech lead}
per discutere delle scelte progettuali ed implementative da me fatte, questo verrà approfondito nel capitolo che segue.

\subsection{Revisioni di progetto}
\label{sub-sect:revisioni-progetto}
La progettazione e lo sviluppo del progetto sono state intervallate da periodici controlli sulla qualità di quanto da me prodotto. Questi controlli erano
effettuati insieme ad un sviluppatore esperto facente parte del gruppo di \emph{tech lead} presente in azienda. Durante questi incrontri tutto il codice sorgente
da me scritto veniva revisionato e ne venivano commentate le caratteristiche e le motivazioni dietro ogni mia scelta. Al termine di questa fase, il prodotto
veniva testato andando a creare un esempio di una linea di produzione in cui potevo effettuare test di carico o osservare i comportamenti del prodotto a
situazioni comuni a cui sottoporsi.

%**************************************************************
\section{Analisi del progetto}
Nonostante abbia sviluppato l'intero applicativo in autonomia, ho seguito tutti i passi adoperati dal gruppo di sviluppo aziendale per la realizzazione
di un progetto interno e quindi non finalizzato ad un cliente specifico.
Questi sono elencati di seguito:
\begin{itemize}
    \item \textbf{Incontro con gli \emph{stakeholder}}: in questa fase viene fatto l’incontro tra il gruppo che andrà a sviluppare il prodotto e tutti 
    gli attori interassati al proggetto. In questa sede vengono discusse varie idea, specificando i requisiti e gli obiettivi che il prodotto dovrà 
    raggiungere;
    \item \textbf{\emph{Scouting} delle tecnologie}: dopo aver acquisito le informazioni utili relative al progetto, il gruppo di lavoro inizia a ricercare 
    le tecnologie che possono risultare più adatte a soddisfare le richieste poste dal committente vagliando anche quelle più sperimentali ed innovative;
    \item \textbf{Studio di fattibilità}: successivamente viene fatto lo studio di fattibilità, in modo tale da poter capire se il prodotto si può sviluppare 
    concretamente. Viene svolta anche un’analisi dei rischi e dei costi, al fine di poter presentare a tutti gli interessati al progetto una chiara panoramica
    di quello che ci si aspetta essere il risultato finale. In questa fase il progetto può essere scartato se considerato non abbastanza vantaggioso;
    \item \textbf{Sviluppo}: se gli attori coinvolti accettano quanto riportato dallo studio di fattibilità, allora si inizia con la fase vera 
    e propria dello sviluppo del prodotto. 
    Il lavoro viene suddiviso in tanti incrementi, come descritto nel modello Agile, così da averlo sempre sotto controllo e poter sempre mostrare 
    al committente i risultati che si stanno ottenendo;
    \item \textbf{\emph{Testing}}: in questa fase il prodotto viene testato mediante specifici test, sia manuali che automatici, per verificarne le 
    funzionalità e assicurarsi che il prodotto abbia soddisfatto le richieste del committente.
\end{itemize}

\subsection{Funzionalità}
\subsection{Requisiti chiave}
\subsection{Casi d'uso}
\subsection{Tecnologie coinvolte}

%Durante la fase di analisi iniziale sono stati individuati alcuni possibili rischi a cui si potrà andare incontro.
%Si è quindi proceduto a elaborare delle possibili soluzioni per far fronte a tali rischi.\\

%\begin{risk}{Performance del simulatore hardware}
%    \riskdescription{le performance del simulatore hardware e la comunicazione con questo potrebbero risultare lenti o non abbastanza buoni da causare il fallimento dei test}
%    \risksolution{coinvolgimento del responsabile a capo del progetto relativo il simulatore hardware}
%    \label{risk:hardware-simulator} 
%\end{risk}

%**************************************************************
\section{Sviluppo delle componenti critiche}
\subsection{Design del modulo}
\subsection{Interfacciamento con librarie native}
\subsection{Ottimizzazione dei tempi di lettura}
\subsection{Linguaggio di dominio specifico}
\subsection{Controllo delle porte GPIO}
\subsection{Tolleranza ai guasti}


%**************************************************************
\section{Risultati raggiunti}
\subsection{Requisiti soddisfatti}
\subsection{Prodotti creati}