% !TEX encoding = UTF-8
% !TEX TS-program = pdflatex
% !TEX root = ../tesi.tex

%**************************************************************
% Sommario
%**************************************************************
\cleardoublepage
\phantomsection
\pdfbookmark{Sommario}{Sommario}
\begingroup
\let\clearpage\relax
\let\cleardoublepage\relax
\let\cleardoublepage\relax

\chapter*{Sommario}

Il presente documento descrive il lavoro svolto durante il periodo di stage, della durata di 320 ore, 
dal laureando Samuele Vanini presso l'azienda Aton S.p.A. \\
L'obbiettivo principale del progetto di stage consisteva nella scrittura di un modulo software 
attuo a comandare un lettore RFID Kathrein in modo che questo potesse operare in autonomia
e tramite l'interfacciamento con una piattaforma middleware proprietaria di Aton, chiamata da ora in poi AMP.
Per fare questo, è stato necessario in primo luogo studiare parte della teoria riguardante la comunicazione
RFID, per poi applicarla alle varie feature offerte dal lettore in esame.
Il passo successivo ha riguardato la progettazione di un'archietettura ad-hoc che andasse ad integrare
la libreria proposta dal produttore con le API del middleware AMP. \\
L'intero documento sarà suddiviso nei quattro capitoli di seguito introdotti:
\begin{itemize}
    \item \hyperref[cap:introduzione]{Primo capitolo}: presenta la realtà aziendale andando a descriverne alcune delle componenti chiave, come le tecnologie
    utilizzate ed il mercato di riferimento.
    \item \hyperref[cap:lo-stage]{Secondo capitolo}: presenta il progetto nel suo insieme, andando a descriverne obiettivi e vincoli.
    \item \hyperref[cap:resoconto-progetto]{Terzo capitolo}: presenta l'archietettura sviluppata per il modulo, illustrando le
        scelte implementative più rilevanti.
    \item \hyperref[cap:conclusioni]{Quarto capitolo}: presenta una valutazione retrospettiva delle attività svolte andando ad analizzare il gap formativo
    tra il bagaglio di conoscenze proposte dal mondo universitario e quello richiesto dal progetto svolto.
\end{itemize}

%\vfill
%
%\selectlanguage{english}
%\pdfbookmark{Abstract}{Abstract}
%\chapter*{Abstract}
%
%\selectlanguage{italian}

\endgroup			

\vfill

