% !TEX encoding = UTF-8
% !TEX TS-program = pdflatex
% !TEX root = ../tesi.tex

%**************************************************************
% Sommario
%**************************************************************
\cleardoublepage
\phantomsection
\pdfbookmark{Sommario}{Sommario}
\begingroup
\let\clearpage\relax
\let\cleardoublepage\relax
\let\cleardoublepage\relax

\chapter*{Sommario}

Il presente documento descrive il lavoro svolto durante il periodo di stage, della durata di 320 ore, 
dal laureando \myName  presso l'azienda Aton S.p.A. \\
L'obbiettivo principale del progetto di stage consisteva nella scrittura di un modulo software 
attuo a comandare un lettore RFID Kathrein in modo che questo potesse operare in autonomia
e tramite l'interfacciamento con una piattaforma middleware proprietaria di Aton, chiamata da ora in poi AMP.
Per fare questo, è stato necessario in primo luogo studiare parte della teoria riguardante la comunicazione
RFID, per poi applicarla alle varie feature offerte dal lettore in esame.
Il passo successivo ha riguardato la progettazione di un'archietettura ad-hoc che andasse ad integrare
la libreria proposta dal produttore con le API del middleware AMP. \\
L'intero documento sarà suddiviso nei quattro capitoli di seguito introdotti:
\begin{itemize}
    \item \hyperref[cap:introduzione]{Primo capitolo}: presenta la realtà aziendale e gli obbiettivi dello stage.
    \item \hyperref[cap:lo-stage]{Secondo capitolo}: presenta la realtà di sviluppo attorno ai dispositivi di IoT, concentrandosi
        sui lettori RFID.
    \item \hyperref[cap:resoconto-progetto]{Terzo capitolo}: presenta l'archietettura sviluppata per il modulo, illustrando alcune delle
        scelte implementative più strategiche
    \item \hyperref[cap:conclusioni]{Quarto capitolo}: presenta una valutazione retrospettiva delle attività svolte 
        e alcune considerazioni finali sui possibili epiloghi del progetto.
\end{itemize}

%\vfill
%
%\selectlanguage{english}
%\pdfbookmark{Abstract}{Abstract}
%\chapter*{Abstract}
%
%\selectlanguage{italian}

\endgroup			

\vfill

